% Options for packages loaded elsewhere
\PassOptionsToPackage{unicode}{hyperref}
\PassOptionsToPackage{hyphens}{url}
\PassOptionsToPackage{dvipsnames,svgnames,x11names}{xcolor}
%
\documentclass[
  letterpaper,
  DIV=11,
  numbers=noendperiod]{scrartcl}

\usepackage{amsmath,amssymb}
\usepackage{iftex}
\ifPDFTeX
  \usepackage[T1]{fontenc}
  \usepackage[utf8]{inputenc}
  \usepackage{textcomp} % provide euro and other symbols
\else % if luatex or xetex
  \usepackage{unicode-math}
  \defaultfontfeatures{Scale=MatchLowercase}
  \defaultfontfeatures[\rmfamily]{Ligatures=TeX,Scale=1}
\fi
\usepackage{lmodern}
\ifPDFTeX\else  
    % xetex/luatex font selection
\fi
% Use upquote if available, for straight quotes in verbatim environments
\IfFileExists{upquote.sty}{\usepackage{upquote}}{}
\IfFileExists{microtype.sty}{% use microtype if available
  \usepackage[]{microtype}
  \UseMicrotypeSet[protrusion]{basicmath} % disable protrusion for tt fonts
}{}
\makeatletter
\@ifundefined{KOMAClassName}{% if non-KOMA class
  \IfFileExists{parskip.sty}{%
    \usepackage{parskip}
  }{% else
    \setlength{\parindent}{0pt}
    \setlength{\parskip}{6pt plus 2pt minus 1pt}}
}{% if KOMA class
  \KOMAoptions{parskip=half}}
\makeatother
\usepackage{xcolor}
\setlength{\emergencystretch}{3em} % prevent overfull lines
\setcounter{secnumdepth}{-\maxdimen} % remove section numbering
% Make \paragraph and \subparagraph free-standing
\makeatletter
\ifx\paragraph\undefined\else
  \let\oldparagraph\paragraph
  \renewcommand{\paragraph}{
    \@ifstar
      \xxxParagraphStar
      \xxxParagraphNoStar
  }
  \newcommand{\xxxParagraphStar}[1]{\oldparagraph*{#1}\mbox{}}
  \newcommand{\xxxParagraphNoStar}[1]{\oldparagraph{#1}\mbox{}}
\fi
\ifx\subparagraph\undefined\else
  \let\oldsubparagraph\subparagraph
  \renewcommand{\subparagraph}{
    \@ifstar
      \xxxSubParagraphStar
      \xxxSubParagraphNoStar
  }
  \newcommand{\xxxSubParagraphStar}[1]{\oldsubparagraph*{#1}\mbox{}}
  \newcommand{\xxxSubParagraphNoStar}[1]{\oldsubparagraph{#1}\mbox{}}
\fi
\makeatother


\providecommand{\tightlist}{%
  \setlength{\itemsep}{0pt}\setlength{\parskip}{0pt}}\usepackage{longtable,booktabs,array}
\usepackage{calc} % for calculating minipage widths
% Correct order of tables after \paragraph or \subparagraph
\usepackage{etoolbox}
\makeatletter
\patchcmd\longtable{\par}{\if@noskipsec\mbox{}\fi\par}{}{}
\makeatother
% Allow footnotes in longtable head/foot
\IfFileExists{footnotehyper.sty}{\usepackage{footnotehyper}}{\usepackage{footnote}}
\makesavenoteenv{longtable}
\usepackage{graphicx}
\makeatletter
\def\maxwidth{\ifdim\Gin@nat@width>\linewidth\linewidth\else\Gin@nat@width\fi}
\def\maxheight{\ifdim\Gin@nat@height>\textheight\textheight\else\Gin@nat@height\fi}
\makeatother
% Scale images if necessary, so that they will not overflow the page
% margins by default, and it is still possible to overwrite the defaults
% using explicit options in \includegraphics[width, height, ...]{}
\setkeys{Gin}{width=\maxwidth,height=\maxheight,keepaspectratio}
% Set default figure placement to htbp
\makeatletter
\def\fps@figure{htbp}
\makeatother

\KOMAoption{captions}{tableheading}
\makeatletter
\@ifpackageloaded{tcolorbox}{}{\usepackage[skins,breakable]{tcolorbox}}
\@ifpackageloaded{fontawesome5}{}{\usepackage{fontawesome5}}
\definecolor{quarto-callout-color}{HTML}{909090}
\definecolor{quarto-callout-note-color}{HTML}{0758E5}
\definecolor{quarto-callout-important-color}{HTML}{CC1914}
\definecolor{quarto-callout-warning-color}{HTML}{EB9113}
\definecolor{quarto-callout-tip-color}{HTML}{00A047}
\definecolor{quarto-callout-caution-color}{HTML}{FC5300}
\definecolor{quarto-callout-color-frame}{HTML}{acacac}
\definecolor{quarto-callout-note-color-frame}{HTML}{4582ec}
\definecolor{quarto-callout-important-color-frame}{HTML}{d9534f}
\definecolor{quarto-callout-warning-color-frame}{HTML}{f0ad4e}
\definecolor{quarto-callout-tip-color-frame}{HTML}{02b875}
\definecolor{quarto-callout-caution-color-frame}{HTML}{fd7e14}
\makeatother
\makeatletter
\@ifpackageloaded{caption}{}{\usepackage{caption}}
\AtBeginDocument{%
\ifdefined\contentsname
  \renewcommand*\contentsname{Table of contents}
\else
  \newcommand\contentsname{Table of contents}
\fi
\ifdefined\listfigurename
  \renewcommand*\listfigurename{List of Figures}
\else
  \newcommand\listfigurename{List of Figures}
\fi
\ifdefined\listtablename
  \renewcommand*\listtablename{List of Tables}
\else
  \newcommand\listtablename{List of Tables}
\fi
\ifdefined\figurename
  \renewcommand*\figurename{Figure}
\else
  \newcommand\figurename{Figure}
\fi
\ifdefined\tablename
  \renewcommand*\tablename{Table}
\else
  \newcommand\tablename{Table}
\fi
}
\@ifpackageloaded{float}{}{\usepackage{float}}
\floatstyle{ruled}
\@ifundefined{c@chapter}{\newfloat{codelisting}{h}{lop}}{\newfloat{codelisting}{h}{lop}[chapter]}
\floatname{codelisting}{Listing}
\newcommand*\listoflistings{\listof{codelisting}{List of Listings}}
\makeatother
\makeatletter
\makeatother
\makeatletter
\@ifpackageloaded{caption}{}{\usepackage{caption}}
\@ifpackageloaded{subcaption}{}{\usepackage{subcaption}}
\makeatother

\ifLuaTeX
  \usepackage{selnolig}  % disable illegal ligatures
\fi
\usepackage{bookmark}

\IfFileExists{xurl.sty}{\usepackage{xurl}}{} % add URL line breaks if available
\urlstyle{same} % disable monospaced font for URLs
\hypersetup{
  pdftitle={7.3 - Using the Central Limit Theorem},
  colorlinks=true,
  linkcolor={blue},
  filecolor={Maroon},
  citecolor={Blue},
  urlcolor={Blue},
  pdfcreator={LaTeX via pandoc}}


\title{7.3 - Using the Central Limit Theorem}
\author{}
\date{}

\begin{document}
\maketitle


\begin{tcolorbox}[enhanced jigsaw, leftrule=.75mm, opacityback=0, breakable, bottomrule=.15mm, colframe=quarto-callout-note-color-frame, arc=.35mm, rightrule=.15mm, left=2mm, toprule=.15mm, colback=white]
\begin{minipage}[t]{5.5mm}
\textcolor{quarto-callout-note-color}{\faInfo}
\end{minipage}%
\begin{minipage}[t]{\textwidth - 5.5mm}

It is important to understand when to use the central limit theorem.

\begin{itemize}
\tightlist
\item
  Find the probability (or percentile) of the mean? CLT for the means
\item
  Find the probability (or percentile) of sum or total? CLT for sums
\item
  Find the probability of an individual value? do \textbf{not} use CLT,
  \emph{use the distribution of its random variable}
\end{itemize}

\end{minipage}%
\end{tcolorbox}

\subsection{Examples of the Central Limit
Theorem}\label{examples-of-the-central-limit-theorem}

\subsubsection{Law of Large Numbers}\label{law-of-large-numbers}

\begin{itemize}
\tightlist
\item
  If you take samples of larger and larger sizes from any population,
  then the mean \(\bar{x}\) of the samples tends to get closer and
  closer to \(\mu\).
\item
  From the central limit theorem, we know as \(n\) gets larger, the
  sample means follow a normal distribution. Recall the standard
  deviation of the sample means: \(\bar{X}=\frac{\sigma}{\sqrt n}\)

  \begin{itemize}
  \tightlist
  \item
    Thus, the larger \(n\) gets, the smaller the standard deviation
    gets.
  \end{itemize}
\end{itemize}

\subsubsection{Example 1}\label{example-1}

A study involving stress is conducted among students on a college
campus. The stress scores follow a uniform distribution with the lowest
score equal to one and the highest equal to five. Using a sample of 75
students find:

\begin{enumerate}
\def\labelenumi{\alph{enumi}.}
\item
  the probability that the \emph{mean stress score} for the 75 students
  is less than 2
\item
  The 90th percentile for the \emph{mean stress score} for the 75
  students
\item
  the probability that the \emph{total stress score} is less than 200
\item
  the 90th percentile for the \emph{total stress score} for the 75
  students
\end{enumerate}

\begin{itemize}
\tightlist
\item
  Let \$X = \$ one stress score
\item
  Let \(\bar{X} =\) mean stress score of sample on parts (a) and (b)
\item
  Let \(\Sigma{X} =\) sum of stress scores of sample on parts (c) and
  (d)
\item
  Recall for uniform distribution \(X \sim U(a,b)\) that
  \(\mu=\frac{a+b}{2}, \sigma=\sqrt{\frac{(b-a)^2}{12}}\)

  \begin{itemize}
  \tightlist
  \item
    \(\mu_X = \frac{1+5}{2}=3\)
  \item
    \(\sigma_X = \sqrt{\frac{(b-a)^2}{12}}=\sqrt{\frac{(5-1)^2}{12}}=1.15\)\\
  \end{itemize}
\item
  Thus \(\bar{X} \sim N(3, \frac{1.15}{\sqrt 75})\)
\item
  And \(\Sigma{X} \sim N(75\cdot 3, \sqrt 75 \cdot 1.15)\)
\end{itemize}

\textbf{Solving (a)}\\
\(P(\bar x < 2)\):
\texttt{pnorm(q\ =\ 2,\ mean\ =\ 3,\ sd\ =\ 1.15/sqrt(75))}
\(=2.5\times10^{-14} \approx 0\)

\textbf{Solving (b)}\\
\(P(\bar{x}<k)=0.90\):
\texttt{qnorm(p\ =\ .9,\ mean\ =\ 3,\ sd\ =\ 1.15/sqrt(75))} \(=3.17\)

\textbf{Solving (c)}\\
\(P(\Sigma x < 200)\):
\texttt{pnorm(q\ =\ 200,\ mean\ =\ 75*3,\ sd\ =\ sqrt(75)*1.15)}
\(=.006\)

\textbf{Solving (d)} \(P(\Sigma{x} < k)=0.90\):
\texttt{qnorm(p\ =\ .9,\ mean\ =\ 75*3,\ sd\ =\ sqrt(75)*1.15)}
\(=237.8\)

\subsubsection{Question 1}\label{question-1}

Use the information from Example 1, but with a sample size of 55 to
answer the following:

\begin{enumerate}
\def\labelenumi{\alph{enumi}.}
\item
  Find \(P(\bar x < 7)\)
\item
  Find \(P(\Sigma x  > 170)\)
\item
  Find the 80th percentile for the mean of 55 scores
\item
  Find the 85th percentile for the sum off 55 scores
\end{enumerate}

\vspace{6cm}

\subsubsection{Question 2}\label{question-2}

Based on data from the National Health Survey, women between the ages of
18 and 24 have an average systolic blood pressure (in mm Hg) of 114.8
with a standard deviation of 13.1. Systolic blood pressure for women
between the ages of 18 and 24 follows a normal distribution.

\begin{enumerate}
\def\labelenumi{\alph{enumi}.}
\item
  If one woman from this population is randomly selected, find the
  probability that her systolic blood pressure is greater than 120.
\item
  If 40 women from this population are randomly selected, find the
  probability that their mean systolic blood pressure is greater than
  120.
\end{enumerate}

\vspace{6cm}

\subsubsection{Question 3}\label{question-3}

According to data from an aerospace company, the 757 airliner carries
200 passengers and has doors with a mean height of 72 inches. Assume for
a certain population of men we have a mean of 69 inches and a standard
deviation of 2.8 inches.

\begin{itemize}
\tightlist
\item
  Let \(X = \blank\)
\item
  Let \(\bar X = \blank\)
\item
  \(X \sim N(\blank, \blank)\)\\
\item
  \(\bar X \sim N(\blank, \blank)\)
\end{itemize}

\begin{enumerate}
\def\labelenumi{\alph{enumi}.}
\tightlist
\item
  What mean doorway height would allow 95 percent of men to enter the
  aircraft without bending?
\end{enumerate}

\(P(x > k) = 0.95\)

\vspace{2cm}

\begin{enumerate}
\def\labelenumi{\alph{enumi}.}
\setcounter{enumi}{1}
\tightlist
\item
  Assume that half of the 200 passengers are men. What mean doorway
  height satisfies the condition that there is a 0.95 probability that
  this height is greater than the mean height of 100 men?
\end{enumerate}

\(P(\bar{x} > k) = 0.95\)

\vspace{2cm}

\begin{enumerate}
\def\labelenumi{\alph{enumi}.}
\setcounter{enumi}{2}
\tightlist
\item
  For engineers designing the 757, which result is more relevant: the
  height from part (a) or part (b)? Why?
\end{enumerate}

\vspace{2cm}




\end{document}
