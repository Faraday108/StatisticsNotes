% Options for packages loaded elsewhere
\PassOptionsToPackage{unicode}{hyperref}
\PassOptionsToPackage{hyphens}{url}
\PassOptionsToPackage{dvipsnames,svgnames,x11names}{xcolor}
%
\documentclass[
  letterpaper,
  DIV=11,
  numbers=noendperiod]{scrartcl}

\usepackage{amsmath,amssymb}
\usepackage{iftex}
\ifPDFTeX
  \usepackage[T1]{fontenc}
  \usepackage[utf8]{inputenc}
  \usepackage{textcomp} % provide euro and other symbols
\else % if luatex or xetex
  \usepackage{unicode-math}
  \defaultfontfeatures{Scale=MatchLowercase}
  \defaultfontfeatures[\rmfamily]{Ligatures=TeX,Scale=1}
\fi
\usepackage{lmodern}
\ifPDFTeX\else  
    % xetex/luatex font selection
\fi
% Use upquote if available, for straight quotes in verbatim environments
\IfFileExists{upquote.sty}{\usepackage{upquote}}{}
\IfFileExists{microtype.sty}{% use microtype if available
  \usepackage[]{microtype}
  \UseMicrotypeSet[protrusion]{basicmath} % disable protrusion for tt fonts
}{}
\makeatletter
\@ifundefined{KOMAClassName}{% if non-KOMA class
  \IfFileExists{parskip.sty}{%
    \usepackage{parskip}
  }{% else
    \setlength{\parindent}{0pt}
    \setlength{\parskip}{6pt plus 2pt minus 1pt}}
}{% if KOMA class
  \KOMAoptions{parskip=half}}
\makeatother
\usepackage{xcolor}
\setlength{\emergencystretch}{3em} % prevent overfull lines
\setcounter{secnumdepth}{-\maxdimen} % remove section numbering
% Make \paragraph and \subparagraph free-standing
\makeatletter
\ifx\paragraph\undefined\else
  \let\oldparagraph\paragraph
  \renewcommand{\paragraph}{
    \@ifstar
      \xxxParagraphStar
      \xxxParagraphNoStar
  }
  \newcommand{\xxxParagraphStar}[1]{\oldparagraph*{#1}\mbox{}}
  \newcommand{\xxxParagraphNoStar}[1]{\oldparagraph{#1}\mbox{}}
\fi
\ifx\subparagraph\undefined\else
  \let\oldsubparagraph\subparagraph
  \renewcommand{\subparagraph}{
    \@ifstar
      \xxxSubParagraphStar
      \xxxSubParagraphNoStar
  }
  \newcommand{\xxxSubParagraphStar}[1]{\oldsubparagraph*{#1}\mbox{}}
  \newcommand{\xxxSubParagraphNoStar}[1]{\oldsubparagraph{#1}\mbox{}}
\fi
\makeatother


\providecommand{\tightlist}{%
  \setlength{\itemsep}{0pt}\setlength{\parskip}{0pt}}\usepackage{longtable,booktabs,array}
\usepackage{calc} % for calculating minipage widths
% Correct order of tables after \paragraph or \subparagraph
\usepackage{etoolbox}
\makeatletter
\patchcmd\longtable{\par}{\if@noskipsec\mbox{}\fi\par}{}{}
\makeatother
% Allow footnotes in longtable head/foot
\IfFileExists{footnotehyper.sty}{\usepackage{footnotehyper}}{\usepackage{footnote}}
\makesavenoteenv{longtable}
\usepackage{graphicx}
\makeatletter
\newsavebox\pandoc@box
\newcommand*\pandocbounded[1]{% scales image to fit in text height/width
  \sbox\pandoc@box{#1}%
  \Gscale@div\@tempa{\textheight}{\dimexpr\ht\pandoc@box+\dp\pandoc@box\relax}%
  \Gscale@div\@tempb{\linewidth}{\wd\pandoc@box}%
  \ifdim\@tempb\p@<\@tempa\p@\let\@tempa\@tempb\fi% select the smaller of both
  \ifdim\@tempa\p@<\p@\scalebox{\@tempa}{\usebox\pandoc@box}%
  \else\usebox{\pandoc@box}%
  \fi%
}
% Set default figure placement to htbp
\def\fps@figure{htbp}
\makeatother

\KOMAoption{captions}{tableheading}
\makeatletter
\@ifpackageloaded{tcolorbox}{}{\usepackage[skins,breakable]{tcolorbox}}
\@ifpackageloaded{fontawesome5}{}{\usepackage{fontawesome5}}
\definecolor{quarto-callout-color}{HTML}{909090}
\definecolor{quarto-callout-note-color}{HTML}{0758E5}
\definecolor{quarto-callout-important-color}{HTML}{CC1914}
\definecolor{quarto-callout-warning-color}{HTML}{EB9113}
\definecolor{quarto-callout-tip-color}{HTML}{00A047}
\definecolor{quarto-callout-caution-color}{HTML}{FC5300}
\definecolor{quarto-callout-color-frame}{HTML}{acacac}
\definecolor{quarto-callout-note-color-frame}{HTML}{4582ec}
\definecolor{quarto-callout-important-color-frame}{HTML}{d9534f}
\definecolor{quarto-callout-warning-color-frame}{HTML}{f0ad4e}
\definecolor{quarto-callout-tip-color-frame}{HTML}{02b875}
\definecolor{quarto-callout-caution-color-frame}{HTML}{fd7e14}
\makeatother
\makeatletter
\@ifpackageloaded{caption}{}{\usepackage{caption}}
\AtBeginDocument{%
\ifdefined\contentsname
  \renewcommand*\contentsname{Table of contents}
\else
  \newcommand\contentsname{Table of contents}
\fi
\ifdefined\listfigurename
  \renewcommand*\listfigurename{List of Figures}
\else
  \newcommand\listfigurename{List of Figures}
\fi
\ifdefined\listtablename
  \renewcommand*\listtablename{List of Tables}
\else
  \newcommand\listtablename{List of Tables}
\fi
\ifdefined\figurename
  \renewcommand*\figurename{Figure}
\else
  \newcommand\figurename{Figure}
\fi
\ifdefined\tablename
  \renewcommand*\tablename{Table}
\else
  \newcommand\tablename{Table}
\fi
}
\@ifpackageloaded{float}{}{\usepackage{float}}
\floatstyle{ruled}
\@ifundefined{c@chapter}{\newfloat{codelisting}{h}{lop}}{\newfloat{codelisting}{h}{lop}[chapter]}
\floatname{codelisting}{Listing}
\newcommand*\listoflistings{\listof{codelisting}{List of Listings}}
\makeatother
\makeatletter
\makeatother
\makeatletter
\@ifpackageloaded{caption}{}{\usepackage{caption}}
\@ifpackageloaded{subcaption}{}{\usepackage{subcaption}}
\makeatother

\usepackage{bookmark}

\IfFileExists{xurl.sty}{\usepackage{xurl}}{} % add URL line breaks if available
\urlstyle{same} % disable monospaced font for URLs
\hypersetup{
  pdftitle={8.1 - A Single Population Mean Using the Normal Distribution},
  colorlinks=true,
  linkcolor={blue},
  filecolor={Maroon},
  citecolor={Blue},
  urlcolor={Blue},
  pdfcreator={LaTeX via pandoc}}


\title{8.1 - A Single Population Mean Using the Normal Distribution}
\author{}
\date{}

\begin{document}
\maketitle


\subsection{Calculating a Confidence Interval
(CI)}\label{calculating-a-confidence-interval-ci}

\begin{itemize}
\tightlist
\item
  Goal is to find confidence interval for \emph{population} mean.
\item
  Constructed from point estimate \(\bar x\), the sample mean

  \begin{itemize}
  \tightlist
  \item
    ``What do we think \(\mu\) probably is?
  \end{itemize}
\item
  And the ``margin of error'' \textbf{EBM} or ``error bound of sample
  mean''

  \begin{itemize}
  \tightlist
  \item
    ``How far off could our estimate be?''
  \end{itemize}
\end{itemize}

\begin{tcolorbox}[enhanced jigsaw, breakable, leftrule=.75mm, toprule=.15mm, bottomrule=.15mm, left=2mm, arc=.35mm, opacityback=0, colframe=quarto-callout-note-color-frame, colback=white, rightrule=.15mm]
\begin{minipage}[t]{5.5mm}
\textcolor{quarto-callout-note-color}{\faInfo}
\end{minipage}%
\begin{minipage}[t]{\textwidth - 5.5mm}

\vspace{-3mm}\textbf{Confidence Interval}\vspace{3mm}

Form of the confidence interval:

\[
(\bar x - EBM, \bar x + EBM)
\]

\end{minipage}%
\end{tcolorbox}

\begin{itemize}
\tightlist
\item
  The margin of error \emph{EBM} depends on the confidence level (CL).\\
\item
  \textbf{Confidence Level (CL)}: ``What is the probability that \(\mu\)
  is within the confidence interval?

  \begin{itemize}
  \tightlist
  \item
    \(CL = 1-\alpha\)
  \end{itemize}
\item
  Related probability \textbf{alpha}: ``What is the probability that
  \(\mu\) is outside the confidence interval?'' \(\alpha=1-CL\)
\end{itemize}

\begin{figure}

\centering{

\pandocbounded{\includegraphics[keepaspectratio]{8.1_files/figure-pdf/fig-plot1-1.pdf}}

}

\caption{\label{fig-plot1}}

\end{figure}%

\subsubsection{To construct confidence interval for population
mean}\label{to-construct-confidence-interval-for-population-mean}

There are two methods to go about this with what we know so far,
recalling that means are distributed
\(N\sim(\mu_X, \frac{\sigma_X}{\sqrt n})\).

\begin{tcolorbox}[enhanced jigsaw, breakable, leftrule=.75mm, toprule=.15mm, bottomrule=.15mm, left=2mm, arc=.35mm, opacityback=0, colframe=quarto-callout-note-color-frame, colback=white, rightrule=.15mm]
\begin{minipage}[t]{5.5mm}
\textcolor{quarto-callout-note-color}{\faInfo}
\end{minipage}%
\begin{minipage}[t]{\textwidth - 5.5mm}

\vspace{-3mm}\textbf{Confidence Interval: \(z\)-scores}\vspace{3mm}

\begin{enumerate}
\def\labelenumi{\arabic{enumi}.}
\tightlist
\item
  Given a confidence level, we can use \emph{z}-scores to find the EBM
  as:

  \begin{itemize}
  \tightlist
  \item
    \(\text{EBM}=z_{\alpha/2} \cdot \frac{\sigma_X}{\sqrt n}\)
  \item
    where \(z_{\alpha/2}\) is the ``critical \emph{z}-score'' that puts
    \textbf{CL} area in the middle, \(\alpha/2\) to the right, and
    \(\alpha/2\) to the left.
  \item
    Recall that this is finding how far a value is from the middle of a
    normal distribution.
  \item
    Can be found with \texttt{qnorm(1-alpha/2)}
  \item
    Then the CI is \(\bar x - \text{EBM}, \bar x + \text{EBM}\)
  \end{itemize}
\end{enumerate}

\end{minipage}%
\end{tcolorbox}

\begin{tcolorbox}[enhanced jigsaw, breakable, leftrule=.75mm, toprule=.15mm, bottomrule=.15mm, left=2mm, arc=.35mm, opacityback=0, colframe=quarto-callout-note-color-frame, colback=white, rightrule=.15mm]
\begin{minipage}[t]{5.5mm}
\textcolor{quarto-callout-note-color}{\faInfo}
\end{minipage}%
\begin{minipage}[t]{\textwidth - 5.5mm}

\vspace{-3mm}\textbf{Confidence Interval: \texttt{qnorm}}\vspace{3mm}

\begin{enumerate}
\def\labelenumi{\arabic{enumi}.}
\setcounter{enumi}{1}
\tightlist
\item
  Given a confidence level, can use \texttt{qnorm} to find the upper and
  lower values directly.

  \begin{itemize}
  \tightlist
  \item
    Lower bound: \texttt{qnorm(alpha/2,\ mean,\ sd/sqrt(n))}
  \item
    Upper bound: \texttt{qnorm(1-alpha/2,\ mean,\ sd/sqrt(n))}
  \end{itemize}
\end{enumerate}

\end{minipage}%
\end{tcolorbox}

\subsubsection{Example 1}\label{sec-E1}

Suppose scores on exams in statistics are normally distributed with an
unknown population mean and a population standard deviation of 3 points.
A random sample of 36 scores is taken and gives a sample mean of 68.
Find a 90\% confidence interval estimate for the true (population) mean
exam score.

\paragraph{\texorpdfstring{Method 1:
\emph{z}-scores}{Method 1: z-scores}}\label{method-1-z-scores}

\begin{itemize}
\tightlist
\item
  We need \(\bar x\) and \textbf{EBM}

  \begin{itemize}
  \tightlist
  \item
    \(\bar x=\blank\)
  \item
    \(\sigma_X = \blank\)
  \item
    \(n = \blank\)
  \item
    \(\text{EBM}=z_{\alpha/2} \cdot \frac{\sigma_X}{\sqrt n}\)

    \begin{itemize}
    \tightlist
    \item
      \(z_{\alpha/2}=z_{(1-\blank)/2}=z_{\blank}\)=\texttt{qnorm(1-\ \ \ \ )}=\(\blank\)
    \item
      \(\sigma_X=\blank\)
    \item
      \(\sqrt{n}=\blank\)
    \item
      \(\text{EBM}=\blank \cdot \frac{\phantom{00000}}{\phantom{00000}}=\blank\)
    \end{itemize}
  \item
    Lower = \(\bar x - \text{EBM}=\blank\)
  \item
    Upper = \(\bar x + \text {EBM}=\blank\)
  \item
    The 90\% Confidence Interval for the mean score is
    \((\blank,\blank)\)
  \end{itemize}
\end{itemize}

\paragraph{Method 2: Directly with
technology}\label{method-2-directly-with-technology}

\begin{itemize}
\item
  Lower: \texttt{qnorm(alpha/2,\ mean,\ sd/sqrt(n))} =
  \texttt{qnorm(\ \ \ ,\ \ \ ,\ \ \ \ \ )} =
\item
  Upper: \texttt{qnorm(1-alpha/2,\ mean,\ sd/sqrt(n))} =
  \texttt{qnorm(\ \ \ ,\ \ \ ,\ \ \ \ \ )} =
\item
  The 90\% Confidence Interval for the mean score is \((\blank,\blank)\)
\end{itemize}

\subsubsection{Question 1}\label{question-1}

Suppose average pizza delivery times are normally distributed with an
unknown population mean and a population standard deviation of 6
minutes. A random sample of 28 pizza delivery restaurants is taken and
has a sample mean delivery time of 36 minutes.

Find a 90\% confidence interval estimate for the population mean
delivery time.

\vspace{6cm}

\subsubsection{Example 2}\label{sec-E2}

What happens if we change the confidence level? Change the original
problem in \hyperref[sec-E1]{Example 1} by using a 95\% confidence
level.

\begin{itemize}
\tightlist
\item
  We need \(\bar x\) and \textbf{EBM}

  \begin{itemize}
  \tightlist
  \item
    \(\bar x=68\)
  \item
    \(\sigma_X = 3\)
  \item
    \(\text{EBM}=z_{\alpha/2} \cdot \frac{\sigma_X}{\sqrt n}\)

    \begin{itemize}
    \tightlist
    \item
      \(z_{\alpha/2}=z_{(1-\blank)/2}=z_{\blank}\)=\texttt{qnorm(1-\ \ \ \ \ )}=\(\blank\)
    \item
      \(\sigma_X=\blank\)
    \item
      \(\sqrt{n}=\blank\)
    \item
      \(\text{EBM}=\blank \cdot \frac{\phantom{00000}}{\phantom{00000}}=\blank\)
    \end{itemize}
  \item
    Lower = \(\bar x - \text{EBM}=\blank\)
  \item
    Upper = \(\bar x + \text {EBM}=\blank\)
  \item
    The 95\% Confidence Interval for the mean score is
    \((\blank,\blank)\)
  \end{itemize}
\end{itemize}

\paragraph{Visualize changing confidence
interval}\label{visualize-changing-confidence-interval}

To compare the confidence intervals in \hyperref[sec-E1]{Example 1} and
\hyperref[sec-E2]{Example 1}, we can compare their graphs.

\begin{figure}

\centering{

\pandocbounded{\includegraphics[keepaspectratio]{8.1_files/figure-pdf/fig-plot2-1.pdf}}

}

\caption{\label{fig-plot2}}

\end{figure}%

\begin{figure}

\centering{

\pandocbounded{\includegraphics[keepaspectratio]{8.1_files/figure-pdf/fig-plot3-1.pdf}}

}

\caption{\label{fig-plot3}}

\end{figure}%

Notice that the where the confidence interval is centered does not
change as the confidence level changes, but rather the confidence
interval is wider for higher confidence levels.

\subsubsection{Summary}\label{summary}

\begin{itemize}
\tightlist
\item
  Increasing the confidence level increases the error bound, making the
  confidence interval wider.
\item
  Decreasing the confidence interval decreases the error bound, making
  the confidence interval narrower.
\end{itemize}

\subsection{Working backwards from Confidence
Intervals}\label{working-backwards-from-confidence-intervals}

\begin{itemize}
\tightlist
\item
  When the confidence interval is calculated, we build it from the
  sample mean and error bound.
\item
  Sometimes in a statistical study, only the confidence interval is
  stated.

  \begin{itemize}
  \tightlist
  \item
    In this case, we can work backward to find the error bound and
    sample mean.
  \end{itemize}
\end{itemize}

\begin{tcolorbox}[enhanced jigsaw, breakable, leftrule=.75mm, toprule=.15mm, bottomrule=.15mm, left=2mm, arc=.35mm, opacityback=0, colframe=quarto-callout-note-color-frame, colback=white, rightrule=.15mm]
\begin{minipage}[t]{5.5mm}
\textcolor{quarto-callout-note-color}{\faInfo}
\end{minipage}%
\begin{minipage}[t]{\textwidth - 5.5mm}

\vspace{-3mm}\textbf{Find Sample Mean from CI}\vspace{3mm}

\begin{gather*}
\bar{x} = \frac{\text{upper bound} + \text{lower bound}}{2} \\
\text{or} \\
\bar{x} = \text{upper bound} - \text{EBM}
\end{gather*}

\end{minipage}%
\end{tcolorbox}

\begin{tcolorbox}[enhanced jigsaw, breakable, leftrule=.75mm, toprule=.15mm, bottomrule=.15mm, left=2mm, arc=.35mm, opacityback=0, colframe=quarto-callout-note-color-frame, colback=white, rightrule=.15mm]
\begin{minipage}[t]{5.5mm}
\textcolor{quarto-callout-note-color}{\faInfo}
\end{minipage}%
\begin{minipage}[t]{\textwidth - 5.5mm}

\vspace{-3mm}\textbf{Find EBM from CI}\vspace{3mm}

\begin{gather*}
\text{EBM} = \frac{\text{upper bound} - \text{lower bound}}{2}\\
or\\
\text{EBM} = \text{upper bound} - \bar x \\
\end{gather*}

\end{minipage}%
\end{tcolorbox}

\subsubsection{Example 3}\label{sec-E3}

Suppose that we know that a confidence interval is \((67.18, 68.82)\)
and we want to find the error bound and sample mean.

Calculate the sample mean:

\[
\begin{aligned}
\bar x &= \frac{67.18 + 68.82}{2} \\
&= \blank
\end{aligned}
\]

Calculate the error bound:

\[
\begin{aligned}
EBM &= \frac{68.82-67.18}{2} \\
&= \blank
\end{aligned}
\]

\subsubsection{Question 2}\label{question-2}

Suppose we know that a confidence interval is \((42.12, 47.88)\). Find
the error bound and sample mean.

\vspace{6cm}

\subsection{\texorpdfstring{Calculating the sample size
\(n\)}{Calculating the sample size n}}\label{calculating-the-sample-size-n}

If a researcher desires a specific margin of error, they can use the
error bound formula
(\(\text{EBM}=z_{\alpha/2} \cdot \frac{\sigma_X}{\sqrt n}\)) to
calculate the required sample size.

\begin{tcolorbox}[enhanced jigsaw, breakable, leftrule=.75mm, toprule=.15mm, bottomrule=.15mm, left=2mm, arc=.35mm, opacityback=0, colframe=quarto-callout-note-color-frame, colback=white, rightrule=.15mm]
\begin{minipage}[t]{5.5mm}
\textcolor{quarto-callout-note-color}{\faInfo}
\end{minipage}%
\begin{minipage}[t]{\textwidth - 5.5mm}

\vspace{-3mm}\textbf{Sample Size Formula}\vspace{3mm}

The formula for sample size:

\(n=\frac{z^2\sigma^2}{\text{EBM}^2}\)

\end{minipage}%
\end{tcolorbox}

\subsubsection{Example 4}\label{example-4}

The population standard deviation for the age of Foothill College
students is 15 years. If we want to be 95 percent confident that the
sample mean age is within 2 years of the true population mean, how many
randomly selected Foothill college students must be surveyed?

\begin{itemize}
\tightlist
\item
  Given: \(\sigma = 15\) and \(\text{EBM} = 2\)
\item
  We need \(z_{\alpha/2}=z_.025 = \blank\)

  \begin{itemize}
  \tightlist
  \item
    Use \texttt{qnorm(\ \ \ \ )}
  \end{itemize}
\end{itemize}

Then:

\[
\begin{aligned}
n&=\frac{z^2\sigma^2}{\text{EBM}^2}\\
&=\frac{(\blank)^2 \cdot 15^2}{2^2}\\
&=\blank
\end{aligned}
\]

Then \(n=\blank\), note you should always round up to ensure the sample
size is large enough.

\subsubsection{Question 3}\label{question-3}

The population standard deviation for the height of high school
basketball players is three inches. If we want to be 95 percent
confident that the sample mean height is within one inch of the true
population height, how many randomly selected students must be surveyed?

\vspace{6cm}

\subsection{Homework}\label{homework}

\begin{enumerate}
\def\labelenumi{\arabic{enumi}.}
\item
  Among various ethnic groups, the standard deviation of heights is
  known to be approximately three inches. We wish to construct a 95\%
  confidence interval for the mean height of male Swedes. 48 male Swedes
  are surveyed. The sample mean is 71 inches. The sample standard
  deviation is 2.8 inches.

  \begin{enumerate}
  \def\labelenumii{\alph{enumii}.}
  \item
    \begin{enumerate}
    \def\labelenumiii{\roman{enumiii}.}
    \tightlist
    \item
      \(\bar x = \blank\)
    \item
      \(\sigma = \blank\)
    \item
      \(n = \blank\)
    \end{enumerate}
  \item
    In words, define the random variables \(X\) and \(\bar X\)
  \item
    Which distribution should you use for this problem? Explain your
    choice
  \item
    Construct a 95\% confidence interval for the population mean height
    of male Swedes.

    \begin{enumerate}
    \def\labelenumiii{\roman{enumiii}.}
    \tightlist
    \item
      State the confidence interval\\
    \item
      Sketch the graph\\
    \item
      Calculate the error bound
    \end{enumerate}
  \item
    What will happen to the level of confidence obtained if 1000 male
    Swedes are surveyed instead of 48? Why?
  \end{enumerate}
\end{enumerate}

\vspace{10cm}




\end{document}
